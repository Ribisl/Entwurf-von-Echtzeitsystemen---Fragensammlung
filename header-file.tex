\documentclass[a4paper]{article}

%%%%%%%%%%%%%%%%%%%%%%%%%%%%%%%%%%%% select which language you want to choose and delete the .aux-file once in the project-folder.
\usepackage[ngerman]{babel}
%\usepackage[english]{babel} 
%%%%%%%%%%%%%%%%%%%%%%%%%%%%%%%%%%%%
\usepackage{pstricks}
\usepackage{pst-circ}
\usepackage{pst-plot}
\usepackage{pstricks-add}
\usepackage{amsmath}
\usepackage[utf8]{inputenc}
\usepackage{geometry}
\geometry{a4paper,left=3cm,right=2cm, top=2cm, bottom=2cm}
\usepackage{graphicx}
\usepackage{listings}
\usepackage{caption}
\usepackage{placeins}
\usepackage{hyperref}
\usepackage{pdfpages}
\usepackage{lipsum}
\usepackage{hhline}
\usepackage{circuitikz}
\usepackage{tabularx,multirow,rotating, gensymb}
\usepackage{subcaption}
\usepackage{float}
\usepackage{parskip}
\newcolumntype{C}[1]{>{\centering\arraybackslash}p{#1}}
\setlength{\parindent}{0pt}		% Einrückungen nach Absatz vermeiden

\newcommand\blfootnote[1]{%
	\begingroup
	\renewcommand\thefootnote{}\footnote{#1}%
	\addtocounter{footnote}{-1}%
	\endgroup
}

\definecolor{darkspringgreen}{rgb}{0.09, 0.45, 0.27}	% Farbe für die Kommentare bei Listings
\lstset{
	language= Matlab, 				    % Setzt die Sprache
	basicstyle=\scriptsize\ttfamily, 	% Setzt den Standardstil
	%keywordstyle=\color{red}\bfseries,	% Setzt den Stil für Schlüsselwörter
	identifierstyle=\color{blue},		% Identifier bekommen keine gesonderte formatierung
	commentstyle=\color{darkspringgreen},		% Stil für Kommentare
	stringstyle=\ttfamily, 			% Stil für Strings (gekennzeichnet mit "String")
	breaklines=true, 			% Zeilen werden umgebrochen
	numbers=left, 				% Zeilennummern links
	numberstyle=\tiny, 			% Stil für die Seitennummern
	frame=single, 				% Rahmen
	%backgroundcolor=\color{myGrey}, 	% Hintergrundfarbe
	%caption={Java-Code}, 			% Caption
	tabsize=2				% Größe der Tabulatoren
}